\documentclass[12pt]{article}
\usepackage[hidelinks]{hyperref}    
\usepackage[all]{hypcap}
\usepackage{graphicx}
\usepackage{amsmath}
\graphicspath{{../../_images/}}
\author{Andrea Malvezzi}
\title{\textbf{Logica per l'informatica~-~Relazioni tra funzioni}}
\date{28 Novembre, 2024}
\begin{document}
\maketitle
\pagebreak
\tableofcontents
\pagebreak

\section{L'insieme quoziente}
Utile quando si lavora con gli infiniti per definire la cardinalità degli insiemi.
\\
Per farlo occorre definire nuovi insiemi dove si usano numeri diversi da quelli naturali.
Per fare ciò risultano utili le proprietà delle funzioni: consideriamo una funzione A e una funzione B.
\begin{itemize}
    \item Se l'insieme A risulta iniettivo rispetto al B, allora B ha sicuramente un numero di elementi uguale ad A, ma potrebbe anche averne di più.
    \item Se l'insieme A risulta biiettivo rispetto al B, allora A e B hanno la stessa cardinalità.
\end{itemize}
Quando si considera l'infinito occorre però usare alcune accortezze:
ad esempio l'insieme dei pari ha cardinalità pari all'insieme dei numeri naturali, in quanto si può scrivere una funzione dove ad ogni naturale si associa un numero pari.
Entrambi questi due insiemi avranno quindi cardinalità pari ad \textit{aleph 0} ($\aleph_0$).
\\
Per indicare la cardinalità di un insieme A si usa la scrittura $|A|$, che indica il numero cardinale $[A]$.

\section{Definizione di insieme finito}
Un insieme si dice finito quando $\dots$ non è infinito.

\section{Operazioni tra numeri cardinali}
Avendo due numeri cardinali $x, y$ corrispondenti rispettivamente ad $|A|$ e $|B|$, allora:

\subsection{Minore uguale}
Dire che $x \leq y$ corrisponde a dire che l'insieme A è iniettivo rispetto \\
all'insieme B, ma che l'insieme B potrebbe contenere ulteriori elementi.
\subsubsection{Esempio}
Consideriamo $2 \leq 3$:
\begin{itemize}
    \item $2 = |\{a, b\}|$; $3 = |\{a, b, c\}|$, perciò $2 \neq 3$.
    \item Considerando i due insiemi appena definiti si può notare come vi sia iniettività da A a B, ma B abbia più elementi.
\end{itemize}
\subsection{Teorema di Cantor}


\end{document}