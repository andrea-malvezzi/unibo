\documentclass[12pt]{article}
\usepackage{amssymb}
\usepackage[hidelinks]{hyperref}    
\usepackage[all]{hypcap}
\usepackage{amsmath}
\title{\textbf{Logica per l'informatica\\Dimostrazioni inerenti alla teoria assiomatica}}
\date{08 ottobre 2024}
\author{Andrea Malvezzi}
\begin{document}
\maketitle
\pagebreak
\tableofcontents
\pagebreak
\section{L'unione}
\subsection{Assioma dell'unione}
Dato un insieme di insiemi, esiste l'insieme che ne è l'unione.
\begin{equation}
    \forall F, \exists X, \forall Z. (Z \in X \Leftrightarrow \exists Y, (Y \in F \wedge Z \in Y)) \label{ass:unione}
\end{equation}
Dove F è l'insieme da unire, mentre Z è un elemento dell'unione sse appartiene ad almeno uno degli insiemi contenuti in F (nel nostro assioma, se appartiene ad un insieme Y di F).\\
L'insieme X viene indicato con $\cup F$ oppure $\cup_{Y \in F}Y$.
\subsubsection{Teorema dell'unione binaria}
Quando l'insieme F è formato solamente da due insiemi, allora si riesce a dimostrare il \textbf{teorema dell'unione binaria}.
Formalmente:
\begin{equation}
    \forall A, \forall B, \exists X, \forall Z. (Z \in X \Leftrightarrow Z \in A \vee Z \in B) \label{teo:unione_binaria}
\end{equation}
Ovvero, dati A e B (i due insiemi da unire) esiste un insieme unione X (che indicheremo come $A \cup B$) tale per cui gli elementi Z di X o appartengono ad A, oppure appartengono a B.
\section{Disgiunzione (or)}
\subsection{Regola di introduzione}
Per dimostrare $P \vee Q$ basta dimostrare o P o Q (due regole di introduzione, ma solo una va dimostrata), dichiarando quale delle due si vuole considerare.\\
Ma non c'è sempre una scelta giusta già dall'inizio della dimostrazione. \\
Vediamo il seguente esempio:
\subsubsection{Problema con la regola di introduzione}
Prendiamo come esempio il seguente teorema:
\[
    \forall X, x \in \mathbb{N}. Pari(x) \vee Dispari(x)
\]
Mi aspetterei che questo teorema sia dimostrabile in quanto ogni numero naturale risulterebbe o pari o dispari.\\
Procediamo per step: preso un numero X, devo dimostrare Pari(X) o Dispari(X).\\
Dimostriamo ora che X è pari. Essendo X un numero generico, nel caso in cui riuscissi a dimostrare la parità di X, dimostrerei in realtà che TUTTI i numeri X siano pari.\\
Per dimostrare il teorema su cui stiamo lavorando occorrebbe generare delle ipotesi basandosi su eventuali teoremi ed annunciati precedentemente elencati, per poi arrivare a sbrogliare la confusione generatasi nell'or.
\subsection{Regola di eliminazione}
A differenza della regola di introduzione, che va posticipata quanto possibile, quella di eliminazione va anticipata il più possibile, in quanto non costituisce una scelta ma anzi, ramifica la dimostrazione e permette di lavorare con più ipotesi.
\subsubsection{In Lean}
\begin{verbatim}
we proceed by cases on NOME_P_or_Q to prove CONCLUSIONE
  case or_1
    suppose P(H)
\end{verbatim}
\quad \quad \dots
\begin{verbatim}
  case or_2
    suppose Q(H)
\end{verbatim}
\quad \quad \dots
\section{Teoremi con l'or}
\subsection{Monotonia dell'unione}
\subsubsection{Teorema}
\begin{equation}
    X \subseteq X^1 \Rightarrow X \cup Y \subseteq X^1 \cup Y. \label{teo:monotonia_unione}
\end{equation}
Voglio dimostrare che se $X$ è sottoinsieme di $X^1$ allora $X$ unito a $Y$ è sottoinsieme di $X^1$ unito $Y$.
\subsubsection{Dimostrazione}
Siano $X$ e $X^1$ due insiemi tali che $X \subseteq X^1$, ovvero $\forall Z.Z \in X \Rightarrow Z \in X^1 (H_1)$.\\
Dobbiamo dimostrare $X \cup Y \subseteq X^1 \cup Y$, ovvero $\forall W. W \in X \cup Y \Rightarrow W \in X^1 \cup Y$.\\
Ora so che W appartiene ad un unione, quindi posso sfruttare (\ref{ass:unione}), secondo il quale W sta in X oppure in Y.\\
A questo punto occorre fare una scelta: 
\begin{itemize}
    \item o dimostrare $W \in X$;
    \item oppure dimostrare $W \in Y$;
\end{itemize}
Nel primo caso, so che $W$ è incluso in $X$ e, tramite $H_1$, allora anche in $X^1$, quindi ho dimostrato la premessa iniziale ($W \in X^1 \cup Y$).\\
Nel secondo invece so che se $W$ è incluso in $Y$ allora sicuramente $W \in X^1 \cup Y$.
\subsection{Esempio di teorema direttamente dimostrabile}
\subsubsection{Teorema}
\[
    X \in Y \Rightarrow X \in Y \cup Z
\]
\subsubsection{Dimostrazione}
Siano X, Y e Z degli insiemi t.c. $X \in Y$ (H). Da (H) ho $X \in Y \vee X \in Z$. Quindi, per (\ref{teo:unione_binaria}), si ha $X \in Y \cup Z$.
\section{L'Esiste}
\subsection{Regola di introduzione}
Per dimostrare $\exists X. P(X)$ devo scegliere una X e dimostrare $P(X)$, in quel caso specifico.\\
Nuovamente, dato che occorre fare una scelta, può essere necessario aspettare per selezionare un X significativo e utile alla dimostrazione di un teorema.\\
Inoltre, in questo caso X può essere anche un'intera espressione, non più solamente una variabile.
\subsection{Regola di eliminazione}
Da un'ipotesi o da un risultato intermedio $\exists X.P(X)$ si può procedere nella prova fissando una variabile generica e supponendo che per tale X valga P, per poi usare tale ipotesi H al fine della dimostrazione.
\section{Teoremi con l'esiste}
\subsection{Primo esempio}
\subsubsection{Teorema}
\[
    X \in \cup F \rightarrow \exists U. (U \in F \wedge U \not= \emptyset)
\]
\subsubsection{Dimostrazione}

\end{document}