\documentclass[12pt]{article}
\usepackage[hidelinks]{hyperref}    
\usepackage[all]{hypcap}
\usepackage{xcolor}
\usepackage{amsmath}
\definecolor{goldenrod_yellow}{RGB}{218, 165, 32}
\definecolor{dark_green}{RGB}{116, 148, 121}
\definecolor{exists_pink}{RGB}{255, 132, 212}
\title{\textbf{Logica per l'informatica\\ brutta}}
\date{05 novembre 2024}
\author{Andrea Malvezzi}
\begin{document}
\maketitle
\pagebreak
\tableofcontents
\pagebreak
\section{Definizioni}
Connotazione $\Rightarrow$ stringhe di caratteri prese in esame;\\
Denotazione $\Rightarrow$ un possibile significato di tale stringa;
\section{Programmi per la logica}
Partendo da una connotazione si possono sviluppare alberi ricorsivi (quelli informatici, non quelli di deduzione), dove ogni nodo è un altro albero fino a terminare con i casi base.
\section{Semantica}
\begin{gather*}
    [[\bot]]^{v} = 0 \\
    [[\top]]^{v} = 1 \\
    [[A]]^{v} = v(A) \\
    [[\not A]]^{v} = 1 - [[A]]^{v} \\
    [[A_1 \wedge A_2]]^{v} = min\{[[A_1]]^{v}, [[A_2]]^{v}\} \\
    [[A_1 \vee A_2]]^{v} = max\{[[A_1]]^{v}, [[A_2]]^{v}\} \\
    [[A_1 \Rightarrow A_2]]^{v} = max\{1 - [[A_1]]^{v}, [[A_2]]^{v}\}
\end{gather*}
\end{document}