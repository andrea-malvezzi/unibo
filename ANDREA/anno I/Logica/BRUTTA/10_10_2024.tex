\documentclass[12pt]{article}
\usepackage[hidelinks]{hyperref}    
\usepackage[all]{hypcap}
\usepackage{xcolor}
\usepackage{amsmath}
\definecolor{goldenrod_yellow}{RGB}{218, 165, 32}
\definecolor{dark_green}{RGB}{116, 148, 121}
\definecolor{exists_pink}{RGB}{255, 132, 212}
\title{\textbf{Logica per l'informatica\\Legenda colori slide}}
\date{08 ottobre 2024}
\author{Andrea Malvezzi}
\begin{document}
\maketitle
\pagebreak
\tableofcontents
\pagebreak
Primo grande insieme saranno i numeri naturali, tramite una meta-logica.\\
Tutti i numeri naturali devono essere distinti, e un modo dei tanti per garantire questa proprietà (e conseguentemente anche molte altre utili per implementare eventuali operazioni) è la seguente:
\[ [[0]] := \emptyset \]
\[ [[n + 1]] := [[n]] \cup {[[n]]} \]
Esempio:
\[ [[0]] = \emptyset \]
\[ [[1]] = {\emptyset} \text{$\emptyset$ è l'insieme corrispondente allo 0} \]
\[ [[2]] = {\emptyset, {\emptyset}} \]
\[ [[3]] = {\emptyset, {\emptyset}}, {\emptyset, {\emptyset}} \]
Dove le doppie quadre indicano la "codifica" di un numero in una data maniera: $:=$ indica una definizione, ovvero la "maniera".

\section{Teorema}
Tutti i numeri naturali sono diversi: quindi non ci sono due insiemi uguali.\\
Sentendo uguale, devo pensare al teorema di estensionalità. Supponiamo quindi 1 = 2.\\
Quindi $\forall Z. (Z \in 1 \Leftrightarrow Z \in 2)$ per l'assioma di estensionalità.\\
Poiché 1 $\in$ 2 per l'assioma dell'unione e del singoletto, dato che 1 appartiene al singoletto vuoto, ne deduciamo che $1 \in 1 = \emptyset$.\\
Se fossimo partiti da 4=5, saremmo dovuti partire dal dimostrare che 3 = 4, poi 2 = 3, etc...\\
Quindi ${\emptyset} = 1 \text{ e } 1 = \emptyset$ CONTINUA SLIDE
\subsection{Osservazione}
Per ora abbiam solo dimostrato l'esistenza di insiemi finiti. Occorre un assioma che chiarisca l'esistenza di un insieme infinito, o almeno che definisca l'insieme contenente \textbf{almeno} i numeri naturali, per poi restringerlo a contenere \textbf{solamente} quelli naturali.\\
I numeri naturali son importanti perché permettono di "catturare" nell'insieme di sistemi abilitati quello dell'informatica, in quanto ogni processo converte dati in sequenze di 1 e 0, quindi numeri naturali.\\
\section{Assioma dell'infinito}
Esiste un insieme che contiene almeno (gli encoding di) tutti i numeri naturali. Questo tuttavia non ci sa dire se in tale insieme stanno dentro solamente quelli naturali, in quanto manca il sse.
\[ \exists Y. (\emptyset \in Y \wedge \forall N. (N \in Y \Rightarrow N \cup \{ N \} \in Y)) \]
Per ogni insieme Y, il vuoto è compreso in tale insieme, e per ogni N, N unito all'insieme N (la definizione di prima della codifica dei numeri naturali, FAI ESEMPIO) appartiene ad Y.
\section{Assioma insieme potenza}
Classico esempio di assioma che risulta utile quando si considerano insiemi infiniti.\\
Preso un insieme ne esiste un insieme dei sottoinsiemi.
\[ X,\exists Y,Z.(Z \in Y \Leftrightarrow Z \subseteq X) \]
Indicando Y come $2^x$ o $P(X)$ (insieme potenza o delle parti):
\[ 2^{\{1,2\}} = \{\emptyset, \{ 1 \}, \{2\}, \{\emptyset, 1, 2\}\} \]
Potremmo anche avere come insieme X un insieme infinito, oppure al posto di $2^x$ qualcosa come $2^{2^{2^x}}$.
\section{Assioma di regolarità o di fondazione}
Ogni insieme non vuoto contiene un elemento da cui è disgiunto.\\
Conseguentemente, nessun insieme contiene ricorsivamente sé stesso: questo è interessante in quanto permette di provare a parlare di cardinalità di un insieme e, quando possibile, di misurarla.
\section{Assioma di rimpiazzamento}
Supponiamo di avere un insieme che son sicuro esista. Supponiamo, attraverso formule logiche, di associare ad ogni elemento un altro insieme.
Allora son sicuro che esista un insieme Y contenente tutti i collegamenti. Questo insieme avrà SEMPRE cardinalità minore o uguale a quella dell'insieme di partenza.
\end{document}