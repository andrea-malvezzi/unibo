\documentclass[12pt]{article}
\usepackage[hidelinks]{hyperref}    
\usepackage[all]{hypcap}
\usepackage{graphicx}
\usepackage{amsmath}
\graphicspath{{../../_images/}}
\author{Andrea Malvezzi}
\title{\textbf{Logica per l'informatica~-~Lezione 0}}
\date{20 Settembre, 2024}
\begin{document}
\maketitle
\pagebreak
\tableofcontents
\pagebreak
Platonismo: esiste già la matematica, esistono già i concetti, dobbiamo arrivare a comprenderli e realizzare la loro presenza; \\
Intuizionismo: non esistono costrutti pre-esistenti, è la mente umana a creare e quindi un ente esiste solamente quando siamo in grado di costruirlo con certezza (non c'è l'infinito, etc...);
\section{La semantica intuizionista}
Anche detta semantica dell'evidenza (evidenza = costruzione), della conoscenza diretta (evidenza = conoscenza diretta) o della calcolabilità.
\section{La prova intuizionista}
Per gli informatici è la logica da preferire quando si studia un problema, in quanto si basa anche sulla calcolabilità, quindi la ricerca dell'evidenza tramite programmi. \\
Per loro questa logica si basa sulla concezione che alla base di ogni cosa vi sia un algoritmo.
\section{Evidenza indiretta}
"So che c'è un qualcosa per cui vale una proprietà" non richiede un'evidenza diretta; \\
"So cos'è il qualcosa per cui vale una proprietà" richiede un'evidenza diretta; \\
Quindi: \\
"So che c'è" è la lettura del \textbf{per ogni, allora esiste} della logica classica; il "so cos'è" è il medesimo ma per la logica intuizionista.
\subsection{Esempio}
Teorema: per ogni n, n è pari oppure n non è pari. \\
La dimostrazione classica direbbe che questo è ovvio per EM (o è pari, oppure non lo è). Ma questo non funziona nella logica intuizionista, in quanto essa punta a capire se un n è pari o dispari:
5 è pari? la prova classica non ce lo dice. Occorre procedere per prova intuizionista: \\
Caso 0: 0 = 2 * 0 e quindi 0 è pari. \\
Caso Sm (successore di m): se m è pari, allora esiste k t.c. m = 2 * k e quindi Sm = 2 * k + 1 (e non sarà perciò pari); se invece m non è pari, allora esiste k t.c. m = 2 * k + 1 e Sm = 2 * k + 2 (e sarà quindi pari). \\
Questo è un approcio ricorsivo strutturale: vuoi sapere se 5 è pari? guarda 4. Poi guarda 3. Poi guarda 2. Poi guarda 1. Poi guarda 0. Sapendo che 0 è pari, allora 1 sarà dispari, 2 sarà pari, 3 dispari, 4 pari e 5 dispari. Ora so che 5 non è pari. 
\section{Approcio informatico}
Nella logica intuizionista, $\forall i. \exists o. P(i, o)$: per ogni i, esiste un o per cui c'è una proprietà che lega i ed o, dove i ed o sono rispettivamente input e output. \\
La logica classica si fermerebbe a dirci che esiste un o oppure no, ma non sapremmo qual è. \\
\section{Regole della logica intuizionista}
Nella logica classica della realtà platonica avevamo:
\begin{itemize}
    \item Un enunciato o è vero o è falso;
    \item Un enunciato ha valore immutabile;
\end{itemize}
La logica intuizionista rifiuta entrambe le cose. In essa la verità di un enunciato è determinato solo quando si ha evidenza diretta (si riesce a trovare un algoritmo). Inoltre, la verità di un enunciato può passare dall'essere indeterminato all'avere un valore che è ORA immutabile (scopro almeno un algoritmo o dimostro che non ne esiste uno). \\
Questo comportamento del passare dall'indeterminatezza a una verità immutabile è detto "maniera monotona" e si può pensare come una linea temporale monodirezionale.
\end{document}