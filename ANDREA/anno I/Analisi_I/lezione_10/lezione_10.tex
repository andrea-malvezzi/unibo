\documentclass[12pt]{article}
\usepackage[hidelinks]{hyperref}    
\usepackage[all]{hypcap}
\usepackage{amssymb}
\usepackage{amsmath}
\usepackage{graphicx}
\graphicspath{{../images/}}
\title{\textbf{Analisi Matematica\\Limiti}}
\date{07 Ottobre 2024}
\author{Andrea Malvezzi}
\begin{document}
\maketitle
\pagebreak
\tableofcontents
\pagebreak
\section{Teorema degli zeri}
\begin{center}
    Avendo una funzione continua del tipo:
\end{center}
\[f:[a,b] \rightarrow \mathbb{R}\]
\begin{center}
    dove
\end{center}
\[f(a) \cdot f(b) < 0\]
\begin{center}
    allora:
\end{center}
\begin{equation}
    \exists c \in (a,b) : f(c) = 0 \label{teo:zeri}
\end{equation}
\subsection{Dimostrazione}
Vedi scritture precedenti per dimostrare (dai lucidi).
\subsection{Applicazioni ai polinomi}
Nei polinomi di grado dispari, in base alla tendenza della x la funzione tenderà a più o a meno infinito:
\begin{gather*}
    \lim_{x \to \infty}{f(x)} = \infty\\
    \lim_{x \to -\infty}{f(x)} = -\infty
\end{gather*}
E perciò, pur non sapendo il valore esatto a cui tende la funzione a $\pm \infty$, sappiamo per certo che:
$$
\begin{cases}
    f(x) > 0 \text{, se } x \to \infty\\
    f(x) < 0 \text{, se } x \to -\infty
\end{cases}
$$
E quindi, grazie al teorema degli zeri, sappiamo che nell'intervallo studiato della funzione si ha un punto dove la funzione si annulla.\\
\textbf{Tutte le funzioni dispari hanno quindi almeno una radice del polinomio.}\\
In quelle di grado pari, questo non è garantito a causa della peculiarità dell'esponente pari di trasformare i negativi in positivi. Potrebbe accadere, ma non è sicuro.
\section{Teorema di Weierstrass}
In un intervallo chiuso (che contiene i suoi estremi) e limitato di una funzione continua, allora:
\begin{gather}
    \exists x_0 \in [a, b] : f(x) \leq f(x_0), \forall x \in [a, b] \label{teo:weierstrass_1} \\
    \exists x_1 \in [a, b] : f(x_1) \leq f(x), \forall x \in [a, b] \label{teo:weierstrass_2}
\end{gather}
\begin{center}
    dove
\end{center}
\begin{gather*}
    f(x_0) = \text{ max } f([a, b]) := M\\
    f(x_1) = \text{ min } f([a, b]) := m
\end{gather*}
\subsection{Teorema di Weierstrass riformulato}
Usando in combinazione il teorema degli zeri e quello di Weierstrass è possibile ottenere quanto segue:
\begin{gather*}
    \exists M = max f([a, b])\\
    \exists m = min f([a, b])
\end{gather*}
\begin{center}
    e vale:
\end{center}
\begin{equation}
    f([a, b]) = [m, M] \label{teo:weierstrass_espansione}
\end{equation}
Vedi dimostrazione dai lucidi.
\section{Introduzione alla nozione di derivata}
\subsection{Cenni di geometria analitica}
Il coefficiente angolare corrisponde alla tangente dell'angolo che la retta studiata forma con l'asse delle $x$.\\
Questo coefficiente si calcola nella maniera seguente:
\begin{equation}
    m = \dfrac{y_1 - y_0}{x_1 - x_0} \label{eq:coefficiente_angolare}
\end{equation}
\begin{center}
    e serve a generare una retta $\dots$
\end{center}
\begin{equation}
    y = m \cdot + q \label{eq:retta_generica}
\end{equation}
\begin{center}
    $\dots$ o più di una, quando si parla di fasci di rette:
\end{center}
\begin{equation}
    y - y_0 = m(x - x_0) \label{eq:fascio_rette}
\end{equation}
\subsection{Retta tangente a un'altra retta in un dato punto}
Avendo un $x_0$, allora si ha che la derivata della funzione in quel tal punto è uguale al limite del rapporto incrementale, ovvero:
\begin{equation}
    \dfrac{f(x) - f(x_0)}{x - x_0} \label{eq:rapporto_incrementale}
\end{equation}
\end{document}