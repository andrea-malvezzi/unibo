\documentclass[12pt]{article}
\usepackage[hidelinks]{hyperref}    
\usepackage[all]{hypcap}   
\usepackage{graphicx}
\usepackage{amsmath}
\graphicspath{{../images/}}
\author{Andrea Malvezzi}
\title{\textbf{Architettura degli Elaboratori\\ ISA del processore HACK}}
\date{05 novembre, 2024}
\author{Andrea Malvezzi}
\begin{document}
\maketitle
\pagebreak
\tableofcontents
\pagebreak

\section{Che cos'è l'ISA}
L'ISA è l'interfaccia tra HW e SW.
Costituisce il set di istruzioni con cui opera la CPU e di conseguenza, varia di architettura in architettura.

\section{L'architettura HACK}
L'architettura HACK non segue né la filosofia CISC né quella RISC e, data la sua semplicità, in essa ad ogni istruzione eseguita corrisponde un ciclo di clock.
Qui si usano una RAM (per contenere i dati di un programma in esecuzione) e una ROM (per contenere il programma \textit{stesso}) a 16 bit. \\
Si utilizzano prevalentemente tre registri di memoria: \textbf{A}, \textbf{D} ed \textbf{M}.
Quest'ultimo è un registro particolare, in quanto contiene il valore del registro di memoria RAM attualmente puntato da A. Questo si indica con RAM[A]. \\
Oltre a questi due si usa inoltre il \textbf{Program Counter}, o \textbf{PC}. In esso è contenuto l'indirizzo della prossima istruzione da eseguire e si indica con ROM[PC].

\section{Tipi di istruzione}
\end{document}