\documentclass[12pt]{article}
\usepackage[hidelinks]{hyperref}    
\usepackage[all]{hypcap}   
\usepackage{graphicx}
\usepackage{amsmath}
\usepackage{xcolor}
\graphicspath{{../images/}}
\author{Andrea Malvezzi}
\title{\textbf{Architettura degli Elaboratori~-~Brutta}}
\date{09 Ottobre, 2024}
\author{Andrea Malvezzi}
\begin{document}
\maketitle
\pagebreak
\tableofcontents
\pagebreak
Differenze tra nostri calcolatori e ALU HACK:\\
La ALU dell'Architettura HACK è a 16 bit.\\
No bit di overflow, occorre andare a guardare i bit segno e fixare l'errore a mano.\\
Esercizio per capire meglio: implementare circuito di mul con ALU HACK
\section{Struttura HACK}
\subsection{Adder}
Circuito per fare le somme, prendendo due input (entrambi a 16 bit) e produce un out (a 16 bit) e un bit di riporto.\\
Noi sappiamo implementarlo, la somma effettiva sono xor, il riporto è una and (si ha riporto solo se son entrambi 1).\\
Vedi appunti cartacei per implementazione di Half-adder e poi full-adder a 3 bit con 2 Half-adder.
Ora abbiamo quasi tutti gli strumenti per costruire l'ALU HACK.\\
Nella tabella dei bit di controllo, zx/nx e zy/ny fanno la stessa cosa, ovvero: se z(x/y) (inteso come una AND tra le due variabili), allora poni x=0. Se n(x/y), allora poni x = !x. Se f è vero, fa la OR, altrimenti la AND. Se ng è vero, nega l'output.\\
Vedi schema uso dei control bit nell'architettura Hack da appunti cartacei.\\
La OR da questo circuito si può fare negando i due ingressi e facendo la AND (De Morgan): $!(\overline{x} \cdot \overline{y}) = x + y$.
Per la sottrazione vedi esempio cartaceo. 
FARE SECONDO PROGETTO e riassumere fino a slide 16 la rappresentazione dell'informazione.
\section{Slide 16 rappresentazione informazione}
Numeri troppo grandi -> overflow; troppo piccoli -> underflow.\\
Spesso quindi non potrò rappresentarli, così come anche talvolta numeri tra altri due numeri (tra il 10 e l'11 magari manco 10,00001 etc...).
\subsection{Floating point}
La codifica Floating point si basa sul rappresentare un numero (n) tramite altri due numeri (f -> frazione o "mantissa" ed e -> esponente o "caratteristica").\\
Di norma si normalizza la frazione, ovvero la prima cifra del decimale non può essere 0. Quindi ad esempio, avendo base 2 e in notazione di eccesso 64:
\[
    \colorbox{yellow}{0}1010100,0000000000011011 = 2^{20} \cdot \dots = 432
\]
Ho 11 zeri dopo la virgola, quindi sposto di 11 cifre a sinistra e rimuovo 11 dall'esponente (parte prima della virgola).\\
A proposito di numero prima della virgola, traducendolo otterrei 84, ma essendo in eccesso 64 devo togliere 64 dal risultato, ottenendo 20 (che è l'esponente). Quindi ora so che il nuovo esponente sarà 20 - 11 = 9, che nella codifica dovrò ricordare di incrementare per l'eccesso, quindi 9 + 64 = 73. Inoltre, dopo la virgola partirò dal primo 1 presente nella codifica originale e scriverò zeri dopo, per riempire eventuali spazi (devo comunque avere 16 bit).\\
Quindi la nuova codifica sarà:
\[
    \colorbox{yellow}{0}1001001,1101100000000000 = 2^9 \cdot (1 \cdot 2^{-1} + 1 \cdot 2^{-2} + 1 \cdot 2^{-4} + 1 \cdot 2^{-5}) = 432    
\]
Un altro esempio, sta volta in base 16 (sempre con eccesso 64):
\[
    \colorbox{yellow}{0}1000101,0000 \text{ } 0000 \text{ } 0001 \text{ } 1011 = 16^5 \cdot (1 \cdot 16^{-3} + B \cdot 16^{-4}) = 432
\]
Qui i bit son raggruppati in cifre esadecimali, quindi occorrerà traslare la frazione di 2 cifre a sinistra (due gruppi) e sottrarre 2 all'esponente. L'esponente è 69 - 64 = 5, e quello nuovo sarà quindi 3 (+64 = 67).
\[
    \colorbox{yellow}{0}1000011,0001 \text{ } 1011 \text{ } 0000 \text{ } 0000 = 16^3 \cdot (1 \cdot 16^{-1} + B \cdot 16^{-2}) = 432
\]
Per l'esponente della frazione, prima converti ogni cifra (ogni gruppo in hexa, i singoli 1 e 0 in decimale, etc...) e moltiplicalo per la base elevato per la posizione che tale cifra occupa nella sequenza (al contrario) (in 1000000000000001 decimale, il primo uno sarà $1 \cdot 2^{-1}$, mentre l'ultimo sarà $1 \cdot 2^{-15}$).
\end{document}