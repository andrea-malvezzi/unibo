\documentclass[12pt]{article}
\usepackage[hidelinks]{hyperref}    
\usepackage[all]{hypcap}   
\usepackage{graphicx}
\usepackage{amsmath}
\usepackage{xcolor}
\graphicspath{{../images/}}
\author{Andrea Malvezzi}
\title{\textbf{Architettura degli Elaboratori\\ALU HACK}}
\date{09 Ottobre, 2024}
\author{Andrea Malvezzi}
\begin{document}
\maketitle
\pagebreak
\tableofcontents
\pagebreak
\section{L'architettura HACK}
L'architettura HACK si basa su un ALU che, a differenza di quella nei PC più comuni, lavora su 16 bit, senza bit di overflow (occorre quindi andare a controllare il segno dei risultati per evitare errori).\\
Questa ALU effettua operazioni su due parole da 16 bit in base al valore logico di 6 bit di controllo, per poi trasmettere su un bus di out il risultato ottenuto.
\subsection{}
\end{document}