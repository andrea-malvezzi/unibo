\documentclass[12pt]{article}
\usepackage{amssymb}
\title{\textbf{Analisi Matematica\\L'insieme numerico $\mathbb{R}$}}
\date{23 settembre 2024}
\author{Andrea Malvezzi}
\begin{document}
\maketitle
\pagebreak
\section{$\sqrt{2}$ è rappresentabile come frazione\\($\sqrt{2} \in \mathbb{Q}$)?}
Per rispondere alla domanda presentata, procederemo con una dimostrazione per assurdo.\\
Assumiamo che $\sqrt{2} \in \mathbb{Q}$. Dunque:
\begin{itemize}
    \item $\exists$ m, n $\in \mathbb{N} | \sqrt{2} = m/n$, m.c.d. (m, n) $= 1$
    \item $\sqrt{2} = m/n$
    \item $2 = m^2/n^2$
    \item $m^2 = 2n^2$ 
\end{itemize}
$2n^2$ è pari (ha 2 tra i divisori), perciò anche $m^2$ è pari. Quindi:
\begin{itemize}
    \item $\exists$ $m\textsubscript{1} \in \mathbb{N} | m\textsubscript{1} = 2n$
    \item ($2m\textsubscript{1})^2 = 2n^2$
    \item $2m\textsubscript{1}^2 = n^2$  
\end{itemize}
ma...
\begin{itemize}
    \item $n^2$ è pari, perciò $n$ è pari;
    \item essendo $n$ ed $m$ pari, viene meno la premessa iniziale per cui $m$ e $n$ non hanno divisori in comune.
\end{itemize}
Ed ecco dimostrato \textbf{per assurdo} che $\sqrt{2} \notin \mathbb{Q}$!
\section{Differenze tra $\mathbb{Q}$ ed $\mathbb{R}$}
Anzitutto occorre specificare che $\mathbb{R}$ nasce come completamento di $\mathbb{Q}$. Questo significa che il primo serve a riempire i buchi - i \textbf{vuoti} - lasciati dal secondo nel piano.\\
Questo significa che in $\mathbb{R}$ non ci sono spazi vuoti ed è quindi un insieme che nutre della proprietà di \textbf{continuità}, anche detta di \textbf{completezza}.
\subsection{Formalizzazione della proprietà di continuità}
Assumendo l'esistenza di $\mathbb{R}$, definiamo:
\begin{itemize}
    \item $\emptyset \not= A \subseteq \mathbb{R}$
\end{itemize}
Ora, per studiare più in dettaglio i sottoinsiemi di $\mathbb{R}$, definiamo tre casi:
\begin{enumerate}
    \item un numero appartenente ad $\mathbb{R}$ si dice \textbf{maggiorante} di un sottoinsieme $A$ se:\\
        $\forall a \in A, M \geqslant a, M \in \mathbb{R}$.\\
        Se $A$ ammette un maggiorante all'interno dell'insieme stesso, è detto \textbf{superiormente limitato}.
    \item lo stesso vale per $M\leqslant a$.
    \item Se $A$ ammette sia maggiorante che \textbf{minorante} è detto \textbf{limitato}.
\end{enumerate}
Ovviamente, se $A$ ha un maggiorante o un minorante allora ne ha infiniti.\\
Inoltre nel caso in cui questi siano contenuti nell'insieme stesso prendono il nome di \textbf{MAX} e di \textbf{MIN}.
\section{La proprietà completezza di $\mathbb{R}$}
$\forall$ $\emptyset \not= A \subseteq \mathbb{R}$:
\begin{enumerate}
    \item Se $A$ è superiormente limitato allora $\exists$ $minM\textsubscript{0} (A) =: sup A$, dove $M\textsubscript{0}$ è il minimo maggiorante;
    \item Se $A$ è inferiormente limitato allora $\exists$ $maxM\textsubscript{0} (A) =: inf A$, dove $M\textsubscript{0}$ è il massimo minorante;
\end{enumerate}
Infine, per definizione, il sup e min di un insieme non limitato in uno o in entrambi gli estremi è pari ad infinito (con il rispettivo segno).
\end{document}