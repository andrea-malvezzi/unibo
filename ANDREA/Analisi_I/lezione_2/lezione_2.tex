\documentclass[12pt]{article}
\usepackage{amssymb}
\title{\textbf{Analisi Matematica\\Titolo}}
\date{19 settembre 2024}
\author{Andrea Malvezzi}
\begin{document}
\maketitle
\pagebreak
\section{Definizione di sottoinsieme proprio}
Un insieme $A$ si dice sottoinsieme proprio di $B$ quando vale quanto segue:
\begin{center}
    Se $\emptyset != A \subsetneqq B$
\end{center}
Dove il simbolo $\subsetneqq$ sta per "inclusione stretta", ovvero:
\begin{itemize}
    \item $A \not= B$;
    \item $A \subseteq B$;
\end{itemize}
\subsection{Esempi di sottoinsiemi propri e non propri}
\begin{itemize}
    \item $A$ = $\{1, 4\}$;
    \item $B$ = $\{1, 2, 3, 4\}$;
    \item $C$ = $\{1, 2, 3, 4\}$;
\end{itemize}
Qui, $A$ è un sottoinsieme proprio di $B$ e di $C$. Tuttavia, $B$ non è sottoinsieme proprio di $C$, e viceversa.
\section{Biunivocità e invertibilità di una funzione}
Una funzione si dice biunivoca quando è sia \textit{1-1} che \textit{sv}.\\
Una funzione biunivoca è inoltre \textbf{invertibile}.
\section{Definizione di insieme numerabile}
Un insieme si dice numerabile quando 
\end{document}