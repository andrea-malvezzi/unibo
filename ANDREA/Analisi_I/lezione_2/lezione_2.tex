\documentclass[12pt]{article}
\usepackage[hidelinks]{hyperref}    
\usepackage[all]{hypcap}
\usepackage{amssymb}
\usepackage{amsmath}
\title{\textbf{Analisi Matematica\\Insiemi e cenni di calcolo combinatorio}}
\date{19 settembre 2024}
\author{Andrea Malvezzi}
\begin{document}
\maketitle
\pagebreak
\tableofcontents
\pagebreak
\section{Definizione di sottoinsieme proprio}
Un insieme $A$ si dice sottoinsieme proprio di $B$ quando vale quanto segue:
\begin{equation}
    \text{Se } \emptyset != A \subsetneqq \label{eq:sottoinsieme_proprio}B
\end{equation}
Dove il simbolo $\subsetneqq$ sta per "inclusione stretta", ovvero:
\begin{itemize}
    \item $A \not= B$;
    \item $A \subseteq B$;
\end{itemize}
\subsection{Esempi di sottoinsiemi propri e non propri}
\begin{itemize}
    \item $A$ = $\{1, 4\}$;
    \item $B$ = $\{1, 2, 3, 4\}$;
    \item $C$ = $\{1, 2, 3, 4\}$;
\end{itemize}
Qui, $A$ è un sottoinsieme proprio di $B$ e di $C$. Tuttavia, $B$ non è sottoinsieme proprio di $C$, e viceversa.
\section{Biunivocità e invertibilità di una funzione}
Una funzione si dice biunivoca quando è sia \textit{1-1} che \textit{sv}.\\
Una funzione biunivoca è inoltre \textbf{invertibile}.
\section{Definizione di insieme numerabile}
Un insieme $\mathbb{X}$ si dice numerabile quando esiste una funzione della seguente specie:
\begin{equation}
    g: \mathbb{N} \rightarrow \mathbb{X}, \textit{g è sv} \label{eq:insieme_numerabile}
\end{equation}
\pagebreak
\section{Elementi di calcolo combinatorio}
\subsection{Fattoriale di un numero}
Avendo $\mathbb{N} = \{0, 1, 2, 3, 4, ...\}$, e un $n \in \mathbb{N}$, allora si dice \textbf{fattoriale di $n$} il valore $n!$, ovvero:
\begin{equation}
    n! := 
    \begin{cases} 
        1 \cdot 2 \cdot 3 \cdot \dots \cdot (n-1) \cdot n & \text{se } n \geq 1, \\
        1 & \text{se } n = 0 
        \label{eq:fattoriale}
    \end{cases}
\end{equation}
\subsubsection{Esempio del fattoriale di un numero:}
Prendiamo come esempio il $4!$ (anche detto \textbf{4-fattoriale}).
\[4! = 1 \cdot 2 \cdot 3 \cdot 4 = 24\]
Prendiamo ora invece lo $0!$. Ricordando quanto affermato in (\ref{eq:fattoriale}):
\[0! = 1\]
\subsection{Coefficiente binomiale}
Avendo due numeri tali che $n, m \in \mathbb{N} : m \leq n$, allora si dice \textbf{Coefficiente binomiale}:
\begin{equation}
    \binom{n}{m} := \dfrac{n!}{(n-m)!m!} \label{eq:coefficiente_binomiale}
\end{equation}
Dove $\dfrac{n!}{(n-m)m!}$ corrisponde a una \textbf{combinazione semplice}.
\subsubsection{Esempio di coefficiente binomiale:}
Avendo $n=3 \text{e } m=2$, allora:
\[
    \binom{3}{2} = \dfrac{3!}{2!(3-2)!} = \dfrac{6}{2} = 3
\]
\pagebreak
\subsection{Prima proprietà del coefficiente binomiale}
\begin{equation}
    \binom{n}{k} = \binom{n}{n-k} \label{prop1_coeff_binomiale}
\end{equation}
Ovvero: ad ogni sottoinsieme di $k$ elementi corrisponde un sottoinsieme di $n-k$ elementi, per cui il loro numero è uguale.
\subsubsection{Prova algebrica}
\[
    \binom{n}{n-k} = \dfrac{n!}{[n-(n-k)]!(n-k)!} = \dfrac{n!}{k!(n-k)!} = \binom{n}{k}
\]
Per fornire un esempio concreto:
\[
    \binom{7}{2} = \dfrac{7!}{[7 - (7 - 2)]!(7-2)!} = \dfrac{7!}{5!2!} = \binom{7}{5} = \binom{7}{7-2}
\]
\subsection{Seconda proprietà del coefficiente binomiale}
\begin{equation}
    \binom{n}{k-1} + \binom{n}{k} = \binom{n+1}{k} \label{prop2_coeff_binomiale}
\end{equation}
\end{document}