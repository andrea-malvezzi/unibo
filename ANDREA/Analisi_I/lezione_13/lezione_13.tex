\documentclass[12pt]{article}
\usepackage[hidelinks]{hyperref}    
\usepackage[all]{hypcap}
\usepackage{amssymb}
\usepackage{amsmath}
\usepackage{graphicx}
\graphicspath{{../images/}}
\title{\textbf{Analisi Matematica\\Limiti}}
\date{07 Ottobre 2024}
\author{Andrea Malvezzi}
\begin{document}
\maketitle
\pagebreak
\tableofcontents
\pagebreak
\section{TEOREMA 1}
Questo teorema si prova in modo diverso a seconda di quale delle due ipotesi si prenda come punto di partenza.\\
$P: f \geq 0 \text{ } \forall x \in (a, b)$ \\
$Q: f \text{ è crescente su } (a, b)$ \\
Questo vale anche quando si considera invece di $\geq$ il $\leq$, cambiando ovviamente anche da crescente a decrescente.\\
Quando si ha invece l'assenza dell'uguale nell'enunciato $P$, si parla di (de)crescente stretta.

\end{document}