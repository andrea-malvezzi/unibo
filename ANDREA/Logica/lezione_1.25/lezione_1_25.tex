\documentclass[12pt]{article}
\author{Andrea Malvezzi}
\title{\textbf{Logica per l'informatica~-~Lezione~1.25.\\Formule logiche (cenni)}}
\date{20 Settembre, 2024}
\begin{document}
\maketitle
\pagebreak
\section{Terminologia utile}
Una proposizione equivale ad una frase per cui ha senso chiedersi se valga o meno:
\begin{itemize}
    \item "2 e un numero pari" $\rightarrow$ è una proposizione, 2 potrebbe essere o non essere pari;
    \item "Mangia la mela!" $\rightarrow$ non è una proposizione, la frase ha un solo significato e non vi è nulla da intuire;
    \item "Luca mangia la mela" $\rightarrow$ è una proposizione, Luca potrebbe mangiare la mela come potrebbe non farlo.
\end{itemize}
Nei linguaggi logici formali (un linguaggio con un determinato set di regole, caratteri etc...) le proposizioni sono dette "\textbf{formule logiche}".\\
Una proposizione si può provare a dimostrare mediante una \textbf{prova}, prendendo il nome di "\textbf{enunciato}".\\
Quando questo processo viene portato a termine con successo, si può allora parlare di \textbf{teorema}.\\
MENTRE si svolge una prova, si può cercare di dimostrare una proposizione o si può assumere che questa valga (in seguito a un'ipotesi o previa dimostrazione).\\
Nel primo caso, questa prenderà il nome di \textbf{conclusione}, mentre nel secondo di \textbf{premessa}.\\
Esistono poi infine i \textbf{Lemmi} e i \textbf{Corollari}: i primi sono teoremi non importanti se presi singolarmente ma utili a dimostarne altri, mentre i secondi sono derivati dalla dimostrazione di un altro teorema. 
\section{Il ragionamento ipotetico}
Quando ipotizziamo \textit{p} non stiamo affermando che \textit{p} valga, ma lo supponiamo, ovvero ci limitiamo a considerare le situazioni dove \textit{p} vale.
\subsection*{Esempio di ragionamento ipotetico}
Ipotizziamo che esistano unicorni rosa volanti e che tutti gli unicorni rosa siano femmine.\\
In questo caso si può concludere che esistano femmine volanti, ma questo vale solamente nei mondi dove esistono unicorni rosa etc... 
\section{Sintassi della logica del prim'ordine}
\begin{itemize}
    \item Falso/assurdo $\rightarrow \bot$ (bottom), che non vale mai;
    \item Vero $\rightarrow \top$ (top), che vale sempre;
    \item \textit{P} \textbf{e} \textit{Q}, \textbf{congiunzione} di \textit{P} e \textit{Q} $\rightarrow$ \textit{P} $\land$ \textit{Q}, che vale se valgono sia \textit{P} che \textit{Q};
    \item \textit{P} \textbf{oppure} \textit{Q}, \textbf{disgiunzione} di \textit{P} e \textit{Q} $\rightarrow$ \textit{P} $\lor$ \textit{Q}, che vale se vale almeno una tra \textit{P} e \textit{Q};
    \item \textbf{se} \textit{P} \textbf{allora} \textit{Q}, \textbf{implicazione logica} tra \textit{P} e \textit{Q} $\rightarrow$ \textit{P} $\Rightarrow$ \textit{Q}, vale se \textit{Q} vale sotto l'ipotesi di \textit{P};
    \item \textbf{non} \textit{P} $\rightarrow \neg$\textit{P}, vale se \textit{P} falso;
    \item \textit{P} \textbf{sse} (in inglese \textbf{iff}) \textit{Q}, \textbf{coimplicazione} di \textit{P} e di \textit{Q} $\rightarrow$ \textit{P} $\Leftrightarrow$ \textit{Q}, che vale se \textit{P} $\Rightarrow$ \textit{Q} e \textit{Q} $\Rightarrow$ \textit{P};
    \item \textbf{Per ogni} x P $\rightarrow \forall$ x.\textit{P}, che vale se \textit{P} vale per ogni possibile valore di x;
    \item \textbf{Esiste} x P $\rightarrow \exists$ x.\textit{P}, che vale se \textit{P} vale per almeno un possibile valore di x;  
\end{itemize}
\section{Ordine delle operazioni logiche}
Per ridurre l'uso delle parentesi:
\begin{itemize}
    \item $\neg, >, <, \lor, \land, \Rightarrow, =, \Leftrightarrow$ (hanno la precedenza);
    \item $\land, \lor, \Rightarrow, \Leftrightarrow$ (sono associativi a destra, si risolve da destra a sinistra); 
\end{itemize}
\pagebreak
\section{Connettivi}
Esistono connettivi di diverse \textbf{arietà}: esistono connettivi 0-ari, unari, binari e quanificatori, in base al numero di dati presi in input.\\
Ad esempio, il simbolo $\neg$ è un operatore unario, in quanto riceve in input solamente un valore da negare, mentre l'operatore $\land$ è binario, in quanto riceve in input due dati.
\subsection*{Esempio di utilizzo corretto di connettivi}
$A \land \neg B \lor C \Rightarrow D \Rightarrow B \lor C \lor E$\\
Questa proposizione logica si legge, tenendo a mente l'ordine degli operatori logici, nella maniera seguente:\\
$((A \land \neg B) \land C) \Rightarrow (D \Rightarrow (B \lor (C \lor E)))$.
\end{document}