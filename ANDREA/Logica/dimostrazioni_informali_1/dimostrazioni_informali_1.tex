\documentclass[12pt]{article}
\usepackage{amssymb}
\usepackage[hidelinks]{hyperref}    
\usepackage[all]{hypcap}
\usepackage{amsmath}
\title{\textbf{Logica per l'informatica\\Dimostrazioni inerenti alla teoria assiomatica}}
\date{08 ottobre 2024}
\author{Andrea Malvezzi}
\begin{document}
\maketitle
\pagebreak
\tableofcontents
\pagebreak
\section{Riflessività del $\subseteq$}
\subsection{Teorema}
\begin{equation}
    X \subseteq X \label{teo:riflessività_subseteq}
\end{equation}
\subsection{Dimostrazione}
Sia X un insieme. Dobbiamo dimostrare il teorema (\ref{teo:riflessività_subseteq}), ovvero:
\begin{center}
    $\forall Y.Y \in X \Rightarrow Y \in X$.
\end{center}
Sia Y un insieme tale che Y $\in$ X(H). Debbo dimostrare Y $\in$ X. Ovvio per l'ipotesi H.
\section{Transitività del $\subseteq$}
\subsection{Teorema}
\begin{equation}
    \text{se } X \subseteq Y \text{ e } Y \subseteq Z \text{ allora } X \subseteq Z \label{teo:transitività_subseteq}
\end{equation}
\subsection{Dimostrazione}
Siano X, Y e Z insiemi tali che $X \subseteq Y$, ovvero $\forall A, A \in X \Rightarrow A \in Y (H_1)$ e $Y \subseteq Z$, ovvero $\forall A, A \in Y \Rightarrow A \in Z (H_2)$.\\
Dobbiamo dimostrare $X \subseteq Z$, ovvero $\forall B, B \in X \Rightarrow B \in Z$. Sia B un insieme t.c. $B \in X (H_3)$. Da $H_3$ e $H_1$ ho $B \in Y$. \\
Quindi per $H_2$ ho $B \in Z$.
\pagebreak
\section{Anti-simmetria del $\subseteq$}
\subsection{Teorema}
\begin{equation}
    \text{se } X \subseteq Y \text{ e } Y \subseteq X \text{ allora } X = Y \label{teo:anti-simmetria_subseteq}
\end{equation}
\subsection{Dimostrazione}
Siano X e Y insiemi t.c. $X \subseteq Y$, ovvero $\forall Z, Z \in X \Rightarrow Z \in X, (H_2)$.\\
Dobbiamo dimostrare $X = Y$. Per l'assioma di estensionalità, è sufficiente dimostrre $\forall Z, Z \in X \Leftrightarrow Z \in Y$. Sia Z un insieme. $Z \in X \Rightarrow Z \in Y$ vale per $H_1$ e $Z \in Y \Rightarrow Z \in X$ vale per $H_2$.
\section{Il vuoto è sottoinsieme di qualunque cosa}
\subsection{Teorema}
\begin{equation}
    \emptyset \subseteq X \label{teo:vuoto_subset_tutto}
\end{equation}
\subsection{Dimostrazione}
Sia X un insieme. Dobbiamo dimostrare $\emptyset \subseteq X$, ovvero $\forall Z. Z \in \emptyset \Rightarrow Z \in X$. Sia Z un insieme t.c. $Z \in \emptyset$ (H). Per l'assioma dell'insieme vuoto $Z \in \emptyset$. Quindi per H assurdo e perciò $Z \in X$.
\section{L'intersezione con il vuoto è il vuoto}
\subsection{Teorema}
\begin{equation}
    X \cap \emptyset = \emptyset \label{teo:intersezione_vuoto}
\end{equation}
\subsection{Dimostrazione}
Sia X un insieme. Dobbiamo dimostrare $X \cap \emptyset = \emptyset$.\\
Per il teorema dell'estensionalità passo a dimostrare che $\forall Z. Z \in X \cap \emptyset \Leftrightarrow Z \in \emptyset$. Ora, supponendo che Z sia un insieme, $Z \in \emptyset \Rightarrow Z \in X \cap \emptyset$ lo abbiamo già dimostrato in (\ref{teo:vuoto_subset_tutto}).\\
Dimostriamo quindi l'affermazione opposta, ovvero: $Z \in X \cap \emptyset \Rightarrow Z \in \emptyset$. Supponiamo $Z \in X \cap \emptyset$. Quindi, per il teorema dell'intersezione binaria, $Z \in X (H_1) \text{ e } Z \in \emptyset (H_2)$. Quindi $Z \in \emptyset$.
\section{L'unico sottoinsieme del vuoto è il vuoto}
\subsection{Teorema}
\begin{equation}
    se X \subseteq \emptyset \text{ allora } X = \emptyset \label{teo:vuoto_unico_subset_vuoto}
\end{equation}
\subsection{Dimostrazione}
$X \subseteq \emptyset$ significa $\forall Z, Z \in X \Rightarrow Z \in \emptyset$ (H). Dobbiamo dimostrare $X = \emptyset$.\\
Per l'assioma di estensionalità possiamo ridurci a dimostrare $\forall Z, Z \in X \Leftrightarrow Z \in \emptyset$. Tuttavia, prendendo un insieme Z, è stato provato in precedenza (\ref{teo:vuoto_subset_tutto}) che $Z \in \emptyset \Rightarrow \in X$.\\
Inoltre $Z \in X \Rightarrow Z \in \emptyset$ vale per H.
\end{document}