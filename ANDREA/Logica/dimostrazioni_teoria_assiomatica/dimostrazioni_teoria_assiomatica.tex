\documentclass[12pt]{article}
\usepackage{amssymb}
\usepackage[hidelinks]{hyperref}    
\usepackage[all]{hypcap}
\usepackage{amsmath}
\title{\textbf{Logica per l'informatica\\Dimostrazioni inerenti alla teoria assiomatica}}
\date{08 ottobre 2024}
\author{Andrea Malvezzi}
\begin{document}
\maketitle
\pagebreak
\tableofcontents
\pagebreak
\section{Introduzione}
Inizialmente, per studiare la logica useremo dimostrazioni informali, ovvero prove meno verbose di quelle rigorose e spesso con dettagli mancanti o perfino errate.\\
A questo fine useremo svariati operatori logici, che ora vedremo nel dettaglio e dal punto di vista dell'utilizzo nel linguaggio Lean.
\section{Per ogni }
Simbolo: $\forall$.
\subsection{Regola di introduzione}
Scopo: dimostrare che $\forall x. P(X)$. Per farlo, lo dimostro su un X generico.\\
La conclusione diventa P.
\subsubsection{In Lean}
\begin{center}
    assume x: set. [dimostrazione di P(x)]
\end{center}
\subsection{Regola di eliminazione}
Scopo: da un'ipotesi o un risultato intermedio $\forall x.P(x)$ concludo che la mia ipotesi vale per un x a mia scelta.
\subsubsection{In Lean}
\begin{center}
    by NOME\_IPOTESI we proved CONCLUSIONE (NOME\_RISULTATO\_INTERMEDIO)
\end{center}
Ad esempio, da $\forall A, \emptyset \subseteq A$ si può ricavare $\emptyset \subseteq \emptyset$ sostituendo ad A l'insieme vuoto.
\pagebreak
\section{Implicazione}
Simbolo: $\Rightarrow$.
\subsection{Regola di introduzione}
Scopo: per dimostrare $P \Rightarrow Q$, lo dimostro su un x generico.\\
La conclusione diventa Q.
\subsubsection{In Lean}
\begin{center}
    suppose P as [NOME\_IPOTESI] [dimostrazione di Q]
\end{center}
\subsection{Regola di eliminazione}
Scopo: da un'ipotesi o un risultato intermedio $P \Rightarrow Q$, posso concludere che Q valga.
\subsubsection{In Lean}
\begin{center}
    by [NOME\_IPOTESI\_PQ],[NOME\_IPOTESI\_P] we proved [Q] as [NOME\_RISULTATO\_INTERMEDIO]
\end{center}
\subsection{Regola di eliminazione (variante)}
Scopo: da un'ipotesi o un risultato intermedio $P \Rightarrow Q$, per dimostrare Q mi posso ridurre a dimostrare P.
\subsubsection{In Lean}
\begin{center}
    by [NOME\_IPOTESI] it suffices to prove [NOME\_IPOTESI\_P]
\end{center}
\pagebreak
\section{Coimplicazione SSE}
Simbolo: $\Leftrightarrow$.
\subsection{Regola di introduzione}
Scopo: dimostrare $P \Leftrightarrow Q$, dimostro sia $P \Rightarrow Q$ che $Q \Rightarrow P$.
\subsubsection{In Lean}
\begin{center}
    we split the proof.\\
    \begin{itemize}
        \centering
        \item proof 1
        \item proof 2
    \end{itemize}
\end{center}
\subsection{Regola di eliminazione}
Scopo: l'ipotesi $P \Leftrightarrow Q$ può essere usata sia come un'ipotesi $P \Rightarrow Q$ che come un'ipotesi $Q \Rightarrow P$.
\section{Assurdo}
\subsection{Regola di eliminazione}
Scopo: se ho dimostrato l'assurdo posso concludere qualsiasi cosa.
\subsubsection{In Lean}
\begin{center}
    by [NOME\_ASSURDO] done \\
    oppure posso anche scrivere\\
    by [NOME\_ASSURDO] we proved false
\end{center}
\section{Negazione}
Simbolo: $\neg$.\\
Scopo: $\overline{P}$ è un'abbreviazione per $P \Rightarrow$ assurdo. Quindi per dimostrare $\overline{P}$ si assume che P valga e si dimostra l'assurdo.\\
\textbf{Attenzione:} questo non serve a dimostrare l'assurdo!
\section{Congiunzione}
Simbolo: $\wedge$.
\subsection{Regola di introduzione}
Scopo: per dimostrare $P \wedge Q$ (P \textbf{e} Q) si dimostrano sia P che Q. Da un'ipotesi P e da un'ipotesi Q si ricava $P \wedge Q$.
\subsubsection{In Lean}
\begin{center}
    by conj, NOMEp, NOMEq we proved $P \wedge Q$ as H
\end{center}
\subsection{Regola di eliminazione}
Scopo: un'ipotesi o un risultato intermedio $P \wedge Q$ può essere usato sia come P che come Q. In alternativa, invece di concludere o assumere $P \wedge Q$ (H), si può direttamente concludere P ($H_1$) e Q ($H_2$).
\subsubsection{In Lean}
\begin{center}
    by NOME we proved P as $H_1$ and Q as $H_2$
\end{center}
\section{Abbreviazioni e formule in Lean}
\begin{itemize}
    \item $\forall x \in A.P(x)$ viene usato per indicare $\forall x \in A, P(x) \forall x. (x \in A \Rightarrow P(x))$
    \item $\exists x \in A.P(x)$ viene usato per indicare $\exists x. (x \in A \wedge P(x))$
    \item $by H_1, H_2 \dots H_n$ we proved P(H) anziché elencare tutte le ipotesi
    \item "thus" per fare riferimento all'ultima ipotesi/risultato intermedio
    \item "we need to prove" per esplicitare la conclusione corrente
    \item "done" per indicare che il lettore è in grado di ricostruire la prova per conto suo 
\end{itemize}
\pagebreak
\section{Enunciati e prove}
\begin{itemize}
    \item L'\textbf{enunciato} è ciò che vogliamo dimostrare, ovvero un insieme di ipotesi e di una conclusione;
    \item una \textbf{prova} è una sequenza di passi che ci convince che la conclusione "valga" quando "valgono" le ipotesi;
    \item per convenzione, tutte le variabili non introdotte da un $\forall$ o da un $\exists$ si considerano introdotte da dei $\forall$ all'inizio dell'enunciato;
    \item tutti gli \textbf{assiomi} sono sempre utilizzabili come ipotesi in qualunque momento.
\end{itemize}
\end{document}