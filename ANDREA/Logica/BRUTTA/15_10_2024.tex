\documentclass[12pt]{article}
\usepackage[hidelinks]{hyperref}    
\usepackage[all]{hypcap}
\usepackage{xcolor}
\usepackage{amsmath}
\definecolor{goldenrod_yellow}{RGB}{218, 165, 32}
\definecolor{dark_green}{RGB}{116, 148, 121}
\definecolor{exists_pink}{RGB}{255, 132, 212}
\title{\textbf{Logica per l'informatica\\Brutta}}
\date{15 ottobre 2024}
\author{Andrea Malvezzi}
\begin{document}
\maketitle
\pagebreak
\tableofcontents
\pagebreak
\section{studio delle dimostrazioni}
Come sono fatte / le regole e come usarle (slides deduzione naturale).\\
Deduzione naturale $\rightarrow$ metodo per dimostrare che un'affermazione è conseguenza di certe ipotesi. La struttura si ramificherà sotto alle prove, formando un albero (si parla di alberi di deduzione naturale).\\
Logica proposizionale $\rightarrow$ meno espressiva di quella del primo ordine con cui abbiamo generato l'insieme n dei numeri naturali. Ma anche più semplice, ottimo per imparare prima di ritornare sulla logica del primo ordine. Trova inoltre molte applicazioni nell'informatica, come nei circuiti logici.\\
Si dice proposizionale perché permette di studiare solamente delle proposizioni, quindi no numeri, etc... ergo no PER OGNI oppure ESISTE.\\
Si usano $\top e \bot$ per indicare qualcosa rispettivamente sempre VERO o sempre FALSO. Per tutte le vie di mezzo dove non si sa se qualcosa sia vero o falso, si usano le variabili dette proposizionali.\\
Un ragionamento si può formalizzare, ovvero tradurre in linguaggio logico:\\
se piove porto l'ombrello e se ho l'ombrello non mi bagno.\\
\[P \Rightarrow O \wedge O \Rightarrow \not B\]\\
Vedi slide lavagna like nelle slide di deduzione naturale.\\
$F$ o $G$, o $F$ o $G$, $F$ oppure $G$, $\dots$ non si capisce quale or si intenda (esiste la XOR e la OR). $F \or G$ or inclusivo (detto vel); $(F \or 1G) \or (1F \or G)$ or esclusivo (detto aut).\\
A volte $F,G$ la virgola significa AND, quindi $F \wedge G$. BISOGNA CAPIRE BENE LA FRASE.\\
I logici scrivono gli alberi come in natura (dal basso verso l'alto), ma li leggono come gli informatici (dall'alto verso il basso).\\
Avendo $B,D \wedge A ? A \wedge (B \Rightarrow C) \Rightarrow C$, allora la mia radice (detta $F$) sarà il teorema da dimostrare (ciò che è dopo il $\vdash$ dell'albero, quindi $A \wedge (B \Rightarrow C) \Rightarrow C$).\\
$\gamma \Rightarrow$ insieme delle ipotesi globali a disposizione. Ad ogni linea orizzontale si associa un passaggio, ovvero una regola degli operatori che abbiamo studiato, ad esempio l'eliminazione dell'implica, l'introduzione, etc$\dots$\\
Inoltre in $B,D \wedge A ? A \wedge (B \Rightarrow C) \Rightarrow C$ la parte prima del $\vdash$ dell'albero è detta gamma ($\gamma$), mentre quella dopo è $F$, ovvero la radice. Quindi un albero di deduzione naturale è detto per $\gamma \vdash F$. Il $\vdash$ si legge deriva ($\gamma \vdash F$, da gamma derivo F).\\

\end{document}