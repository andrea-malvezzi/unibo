\documentclass[12pt]{article}
\usepackage[hidelinks]{hyperref}    
\usepackage[all]{hypcap}
\usepackage{amsmath}
\usepackage{graphicx}
\graphicspath{{../../_images/}}
\title{\textbf{Logica per l'informatica\\Lista teoremi e assiomi}}
\date{10 ottobre 2024}
\author{Andrea Malvezzi}
\begin{document}
\maketitle
\pagebreak
\tableofcontents
\pagebreak
\section*{Introduzione}
A seguire una lista di tutti i teoremi e di tutti gli assiomi utili alle dimostrazioni di laboratorio, con casi di utilizzo e spiegazioni approfondite.
\section{Assiomi}
\subsection{Assioma di estensionalità}
Due insiemi sono uguali quando contengono gli stessi elementi:
\begin{equation}
    X,Y.(X = Y \Leftrightarrow \forall Z.(Z \in X \Leftrightarrow Z \in Y)) \label{ass:estensionalità} \tag{ax. 1}
\end{equation}
Questo teorema si utilizza quando occorre indagare sull'uguaglianza tra due insiemi: potrei usarlo sia per mostrare che due insiemi siano uguali, oppure per mostrare che NON lo siano.
\subsubsection{Esempio}
\begin{gather*}
    A = \{1, 2\}\\
    B = \{1\}\\
    A = B? \text{ 1 è presente in entrambi gli insiemi, ma 2 solamente in A.}
\end{gather*}
\subsubsection{Nota Bene}
Essendo questo un assioma "a due direzioni" (un insieme è uguale a un altro se hanno gli stessi elementi E VICEVERSA) in Lean si potrebbe realizzare DUE assiomi (uno per verso, evitando il sse), per semplificarne l'utilizzo:
\begin{center}
    axiom ax\_extensionality1: $\forall A, B. (\forall Z, Z \in A \Leftrightarrow B) \Rightarrow A = B$\\
    axiom ax\_extensionality2: $\forall A, B, A = B \Rightarrow (\forall Z, Z \in A \Leftrightarrow B)$\\
\end{center}
\pagebreak
\subsection{Assioma di separazione o di comprensione}
Questo assioma permette, partendo da un insieme $A$, di formare un sottoinsieme $B$ in modo tale che quest'ultimo soddisfi una certa proprietà $P(X)$.
\begin{equation}
    A, \exists B, x.(x \in B \Leftrightarrow x \in A \wedge P(x)) \label{ass:separazione_comprensione} \tag{ax. 2}
\end{equation}
$B$ equivale al sottoinsieme che vogliamo creare, quindi possiamo scriverlo come:
\begin{equation}
    B = \{x \in A : P(x)\} \label{def:insieme_B_separazione_comprensione}
\end{equation}
Stiamo quindi dicendo che avendo un insieme A, esiste un insieme B che contiene elementi di A tali per cui valga la proprietà specificata come $P(x)$.
\subsubsection{Esempio}
Avendo un insieme $A$ contenente alcuni numeri naturali, voglio creare un insieme $B$ contenente solamente i numeri pari di $A$. Quindi:
\begin{gather*}
    A = \{1, 2, 3, 4, 5, 6, 7\}\\
    P(x) = \text{ "x è pari."}\\
    B = \{x \in A : P(x)\}, \text{ quindi: }\\
    B = \{2, 4, 6\} 
\end{gather*}
\subsection{Assioma dell'insieme vuoto}
Avendo un insieme X vuoto, allora non ci sono elementi contenuti al suo interno:
\begin{equation}
    \exists X, Z.(Z \not\in X) \label{ass:insieme_vuoto} \tag{ax. 3}
\end{equation}
Questo assioma risulta utile nelle dimostrazioni per assurdo.
\subsubsection{Esempio}
L'insieme vuoto comprende se stesso?
\begin{gather*}
    \emptyset \in \emptyset?\\
    \text{ applichiamo l'assioma del vuoto: } \exists X, \forall z. (z \not\in X)\\
    \text{ sapendo che l'insieme vuoto è per definizione: } \forall z. (z \not\in \emptyset)\\
    \text{ abbiamo, sostituendo alla z l'insieme vuoto: } \forall \emptyset. (\emptyset \not\in \emptyset)
\end{gather*}
Abbiamo quindi dimostrato che l'insieme vuoto non contiene sé stesso.
\pagebreak
\subsection{Assioma dell'unione}
Consideriamo un insieme F come una scatola contenente $n$ insiemi che\\
vogliamo unire tra loro.\\
Allora esiste un insieme detto \textbf{insieme unione} $X$ contenente tutti gli elementi dei sottinsiemi di $F$.  \\
Quindi definendo $X = \cup F$, i membri di $\cup F$ saranno tutti i membri dei membri di $F$.\\
Si usa quando si vuole effettuare un'unione tra $n$ insiemi e la sua proposizione è la seguente:
\begin{equation}
    \forall F, \exists X, \forall Z. (Z \in X \Leftrightarrow \exists Y, (Y \in F \wedge Z \in Y)) \label{ass:unione} \tag{ax. 4}
\end{equation}
\subsubsection{Esempio}
Avendo un insieme $A$ contenente $3$ sottoinsiemi che si vogliono unire tra loro, allora:
\begin{gather*}
    A = \{\{1, 2\}, \{3, 4\}, \{5, 6\}\}\\
    X = \cup A = \{z : \exists Y \in A : z \in Y\}\\
    \text{ quindi: } X = \{1, 2, 3, 4, 5, 6\}
\end{gather*}
\subsection{Assioma del singoletto}
Avendo un insieme $X$, esiste un insieme $Y$ indicato come $\{X\}$ per cui un elemento $Z$ appartiene ad $Y$ se e solo se $Z$ è $X$:
\begin{equation}
    X, Z.(Z \in \{X\} \Leftrightarrow Z = X) \label{ass:singoletto} \tag{ax. 5}
\end{equation}
Questo assioma risulta molto utile quando si vuole formare un insieme contenente solamente un elemento, detto \textbf{singoletto}.\\
Si usa inoltre per comporre insiemi più complessi (ma pur sempre finiti), combinando tra loro diversi singoletti.\\
Ad esempio, per formare la coppia $(a, b)$ si potrebbero prima creare $\{a\} \text{ e } \{b\}$, e poi unirli tramite l'assioma dell'unione (\ref{ass:unione}).
\pagebreak
\subsection{Assioma dell'infinito}
Esiste un insieme che contiene ALMENO tutti gli encoding dei numeri meta-naturali. Questo insieme potrebbe quindi contenere anche più elementi di quelli che stiamo cercando.
\begin{equation}
    \exists Y.(\emptyset \in Y \wedge \forall N. (N \in Y \Rightarrow N \cup \{N\} \in Y)) \label{ass:infinito} \tag{ax. 6}
\end{equation}
Il fatto che tale insieme possa contenere anche più elementi di quelli che stiamo cercando è dovuto alla presenza di una implicazione logica invece che di una COimplicazione.\\
L'insieme risultante si potrà poi filtrare mediante Assioma di separazione \eqref{ass:separazione_comprensione} per ridursi gradualmente sempre di più al sottoinsieme dei numeri naturali.
\subsection{Assioma dell'insieme potenza}
Questo assioma risulta particolarmente utile quando si considerano insiemi infiniti.\\
Afferma che preso un insieme di partenza, allora esiste un altro insieme, detto \textbf{insieme potenza}, contenente tutti i suoi sottoinsiemi:
\begin{equation}
    A, \exists Y, z.(z \in Y \Leftrightarrow z \subseteq X) \label{ass:insieme_potenza} \tag{ax. 7}
\end{equation}
$Y$ si può rappresentare come $P(x)$ o come $2^x$ ed è detto \textbf{insieme potenza} o \textbf{insieme delle parti}.
\subsection{Assioma di regolarità o di fondazione}

\section{Teoremi}
\section{Definizioni}
\subsection{Definizione di numeri naturali} \label{subsec:def:numeri_naturali}
Per ogni meta-numero naturale $n$ occorre definire un insieme [[n]] da identificare con tale numero $n$.
Uno dei tanti modi per definire tali numeri, garantendo allo stesso tempo che questi siano distinti (e molte altre cose)è la seguente:
\begin{gather*}
    [[0]] := \emptyset\\
    [[n + 1]] := [[n]] \cup \{[[n]]\}
\end{gather*}
Dove $[[n]]$ indica la codifica del meta-numero naturale $n$, ottenuta unendo l'insieme corrispondente ad $n-1$ a sé stesso.
\subsubsection{Esempio}
\begin{gather*}
    [[0]] = \emptyset\\
    [[1]] = \{\emptyset\}\\
    [[2]] = \{\emptyset, \{\emptyset\}\}\\
    [[3]] = \{\emptyset, \{\emptyset\}, \{\emptyset, \{\emptyset\}\}\}
\end{gather*}
Dove l'insieme corrispondente a $n=0$ si ottiene unendo l'insieme vuoto a sé stesso.
\end{document}