\documentclass[12pt]{article}
\usepackage[hidelinks]{hyperref}    
\usepackage[all]{hypcap}
\usepackage{xcolor}
\definecolor{goldenrod_yellow}{RGB}{218, 165, 32}
\definecolor{dark_green}{RGB}{116, 148, 121}
\definecolor{exists_pink}{RGB}{255, 132, 212}
\title{\textbf{Logica per l'informatica\\Legenda colori slide}}
\date{08 ottobre 2024}
\author{Andrea Malvezzi}
\begin{document}
\maketitle
\pagebreak
\tableofcontents
\pagebreak
\section{Introduzione}
Questa legenda serve a tenere traccia di cosa significano gli art attack di Coen nelle slide.
Ogni colore corrisponde ad un comando Lean e senza una legenda può risultare difficile seguire le lezioni.
Ed ecco perché era \textbf{imperativo} crearne una.
\section{Legenda}
\textbf{N.B.} tutti i colori sono in ordine di apparenza nelle slide, con la prima apparizione alla dodicesima slide della presentazione "Prove in teoria degli insiemi".
\begin{itemize}
    \item \colorbox{yellow}{Giallo} = Introduzione del forall ($\forall_i$);
    \item \colorbox{green}{Verde chiaro} = Introduzione implicazione logica tra P e Q ($\Rightarrow_i$);
    \item \colorbox{magenta}{Magenta} = Passaggio non corrispondente ad un comando logico / espansione di definizione;
    \item \colorbox{pink}{Rosa chiaro} = Eliminazione della negazione ($\bot_e$);
    \item \colorbox{lightgray}{Grigio chiaro} = Eliminazione dell'and ($\wedge_e$);
    \item \colorbox{goldenrod_yellow}{Ocra} = Introduzione dell'and ($\wedge_i$);
    \item \colorbox{dark_green}{Verde scurdo} = Eliminazione dell'or ($\vee_e$);
    \item \colorbox{gray}{Grigio} = Introduzione ipotesi 1/2 dell'or ($\vee_{i_1}$ / $\vee_{i_2}$);
    \item \colorbox{exists_pink}{Rosa} = Eliminazione exists ($\exists_e$);
    \item \colorbox{cyan}{Azzurro} = Introduzione exists ($\exists_i$);
\end{itemize}
\end{document}